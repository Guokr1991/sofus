\begin{align*}
\int_{-\infty }^{\infty }e^{-x^{2}}e^{-ixp}dx &= e^{-p^{2}/4}\int_{-\infty }^{\infty }e^{-(x+ip/2)^{2}}dx\nonumber \\
&=e^{-p^{2}/4}\int_{-\infty +ip/2 }^{\infty+ip/2 }e^{-x^{2}}dx\nonumber \\
&=e^{-p^{2}/4}\int_{-\infty }^{\infty }e^{-x^{2}}dx\nonumber \\
&=\sqrt{\pi }e^{-p^{2}/4}
\end{align*}
The third equation is justified by contour integration since
$e^{-x^{2}}=O(e^{\m\mathrm{Re}(x)^2})$ as
$\|\mathrm{Re}(x)\|\rightarrow \infty $ for bounded $\|
\mathrm{Im}(x)\|$.

The difference of the integrals is the limit of the integral over the
contour
\begin{equation*}
[-R,R]\union {\color{red} R + ip/2[0,1]} \union [R,-R] + i p/2 \union
{\color{red} -R +ip/2[1,0]}
\end{equation*}
as $R\rightarrow\infty$ because the integral over the red pieces
vanishes. Since the there are no singularities inside the contour, the
integral around the contour is 0.

Now simple manipulation of (1) yields
\begin{align*}
\int_{-\infty }^{\infty }e^{-n^{2}(x-m)^{2}}e^{-ipx}dx &= \int_{-\infty }^{\infty }e^{-n^{2}x^{2}}e^{-i(x+m)p}dx\nonumber \\
&=e^{-imp}\int_{-\infty }^{\infty }e^{-n^{2}x^{2}}e^{-ipx}dx\nonumber \\
&=\frac{e^{-imp}}{n}\int_{-\infty }^{\infty }e^{-x^{2}}e^{-ixp/n}dx\nonumber \\
&=\frac{e^{-imp}}{n}\sqrt{\pi }e^{-p^{2}/(4n^{2})}
\end{align*}

\begin{align}
n^{2} &= \frac{1}{2\sigma^{2}}\nonumber \\
\sigma = \frac{1}{\sqrt{2}n}
\end{align}

\begin{align*}
\frac{1}{4n^{2}} = \frac{1}{2\sigma^{2}}\nonumber \\
\sigma = \sqrt{2}n
\end{align*}
